\documentclass[a4paper,10pt]{article}
\usepackage{paper-en}
\usepackage{hyperref}




%\usepackage[notref,notcite,color]{showkeys}


\def\thetitle{Twistings}
\def\theauthors{}

\hypersetup{colorlinks=true,
citecolor=black,
linkcolor=black,
anchorcolor=black,
filecolor=black,
menucolor=black,
urlcolor=black,
pdftitle={\thetitle},
pdfauthor={\theauthors}
}










%\overfullrule=100mm

\begin{document}

%\pagestyle{empty}\renewcommand\includegraphics[2][{}]{}


\title{}
\author{}
\date{}
\maketitle

%\begin{abstract}
%\end{abstract}

\section{The formula}



Let $L$ be a smooth $n$-dimensional submanifold in a Riemannian manifold $M$.
Let us denote by $\T_pL$ and $\N_pL$ the tangent and normal spaces of $L$ at $p$.

Recall that \emph{second fundamental form} $\II$ at $p$ is a symmetric quadratic form on $\T_pL$ with values in $\N_pL$.
It is uniquely defined by the following identity
\[\II(\vec v,\vec v)\df \gamma''_{\vec v}(0),\]
where $\vec v\in \T_pL$ and $\gamma_{\vec v}$ an $L$-geodesic that starts at $p$ with initial velocity vector~$\vec v$.

Given $p\in L$,
denote by $\Sc(p)$, $H(p)$, and $\zh(p)$
the scalar curvature, the mean curvature vector, and the average value of $|\II(\vec u,\vec u)|^2$ at $p$ respectively.

Let us denote by $\widetilde{\Sc}$ the \emph{outer scalar curvature} of $L$ in $M$;
that is, if $e_1,\dots,e_n$ is an orthonormal basis of $\T_pL$, then 
\[\widetilde\Sc(p)=2\cdot \sum_{i<j} K_{ij},\]
where $K_{ij}$ denotes sectional curvature of $M$ in the direction spanned by $e_i$ and~$e_j$.


The following claim gives a better version of \cite[5.B]{gromov1}.

\begin{thm}{Formula}
The following identity
\[\Sc-\widetilde{\Sc}=\tfrac32\cdot |H|^2-\tfrac{n\cdot (n+2)}{2}\cdot\zh\]
holds for any smooth $n$-dimensional submanifold $L$ in a Riemannian manifold.
\end{thm}


\parit{Proof.}
Without loss of generality, we can assume that the ambient manifold is flat;
in particular $\widetilde\Sc=0$.

Assume $\codim L=1$.
Choose $p\in L$;
denote by $k_1,\dots,k_n$ the principal curvatures of $L$ at $p$.
Note that
\[|H|^2= \sum_ik_i^2+2\cdot\sum_{i<j}k_i\cdot k_j.\]
Further, 
\[
n\cdot (n+2)\cdot\zh
=
3\cdot \sum_i k_i^2+2\cdot \sum_{i<j} k_i\cdot k_j.
\]
The last identity follows since $\zh$ is the average value of $\left(\sum_i k_i\cdot x_i^2\right)^2$ on $\SSS^{n-1}$.
One has to take into account that $3$ and $1$
are the average values of  $n\cdot (n+2)\cdot x_i^4$ and $n\cdot (n+2)\cdot x_i^2\cdot x_j^2$ for $i\ne j$ on the unit sphere $\SSS^{n-1}\subset \RR^n$.
Here we assume that $(x_1,\dots,x_n)$ are the standard coordinates in $\RR^n$.


By Gauss formula,
\[\Sc=2\cdot\sum_{i<j}k_i\cdot k_j.\]
It remains to rewrite this formula using the expressions for $|H|^2$ and $\zh$.

If $\codim L =k>1$, then the second fundamental form can be presented as a direct sum of $k$ real-valued quadratic forms $\II_1\oplus\dots\oplus \II_k$;
that is,
\[\II=e_1\cdot\II_1+\dots+e_k\cdot \II_k,\]
where $e_1,\dots, e_k$ is an orthonormal basis of $\N_pL$.
Denote by $\Sc_i$, $H_i$, and $\zh_i$ the values associated with $\II_i$.
From above, we get
\[\Sc_i=\tfrac32\cdot |H_i|^2-\tfrac{n\cdot (n+2)}{2}\cdot\zh_i\]
for each $i$.

Note that 
\begin{align*}
\Sc&=\Sc_1+\dots+\Sc_k,
\\
|H^2|&=|H_1|^2+\dots+|H_k|^2,
\\
\zh&=\zh_1+\dots+\zh_k.
\end{align*}
Whence the general case follows.
\qeds

\section{Embeddings into sphere}

The obtained formula shows that some results in \cite{gromov1,gromov2,gromov3} are exact.
In this and the following section, we will list of some of them.

Let us denote by $\TT^n$ the $n$-dimensional torus --- the smooth manifold diffeomorphic to the product of $n$ circles.
The next statement follows from the formula since any Riemannian metric on the torus has nonpositive scalar curvature at some point.

\begin{thm}{Theorem}
Suppose $\iota\:\TT^n\looparrowright \SSS^q$ is a smooth immersion.
Then 
\[\zh(p)\ge \tfrac{2\cdot (n-1)}{n+2}\]
at some point $p\in \TT^n$.

In particular, there is a tangent direction of $\TT^n$ with normal curvature at least 
\[\kappa_n=\sqrt{\tfrac{2\cdot (n-1)}{n+2}}.\]
\end{thm}

It was shown \cite{gromov1} that there is an isometric embedding of the torus $\TT^n$ with a flat metric that has normal curvature $\kappa_n$ in any direction at any point. 
In particular, the above bound on normal curvature is optimal.
The compression lemma \cite{gromov3}, implies that \textit{any closed smooth manifold is diffeomorphic to a submanifold with normal curvatures at most $\sqrt{2}$ in the unit sphere of sufficiently large dimension.}
\textit{Moreover, the induced Riemannian metric can be chosen to be proportional to any given metric $g$.}
Applying the theorem, we get the following.

\begin{thm}{Corollary}
The bound $\sqrt{2}$ is optimal.
\end{thm}

\section{Embeddings into ball}

Let us denote by $\BB^q$ the unit ball in $q$-dimensional Euclidean space.

\begin{thm}{Question}
Suppose $\iota\:\TT^n\to \BB^q$ is a smooth immersion.
Is it true that $\zh(p)\ge \tfrac{3\cdot n}{n+2}$ at some point $p\in\TT^n$?
\end{thm}

If true, then for $q\gg n^2\gg1$, the optimal asymptotic lower bound on normal curvatures is $\sqrt{3}$.
Playing a bit with the formulas below seems to give an asymptotic lower bound $\sqrt{8/3}$; it is quite close to $\sqrt{3}$ and can be improved a bit further, but the optimal bound requires an extra idea.

Below we answer the question in three cases: $n=2$, $n=4$, and if the induced metric is flat.
Note also that the theorem in the previous section answers the question if the image of $\iota$ lies in the boundary of $\BB^q$.

The following lemma was essentially proved by István Fáry \cite{fary}; see also the survey of Serge Tabachnikov \cite{tabachnikov}.

\begin{thm}{Lemma}
Let $\iota\:\TT^n\to \BB^q$ be a smooth immersion.
Then the average value of $|H|$ is at least $n$.
\end{thm}

\parit{Proof.}
Consider the function $u\:p\mapsto \tfrac12\cdot |\iota(p)|^2$ on $\TT^n$.
Note that $\Delta u=n+ \langle H,\iota\rangle$.
It follows that the average value of $\langle H,\iota\rangle$ is $-n$.
Since $|\iota|\le1$, we get the result.
\qeds

The lemma and formula imply the following.

\begin{thm}{Proposition}
Let $L$ be a flat closed $n$-dimensional submanifold in $\BB^q$.
Then the average value of $\zh$ on $L$ is at least  $\tfrac{3\cdot n}{n+2}$.
\end{thm}

\begin{thm}{Theorem}
Suppose $\iota\:\TT^2\to \BB^q$ is a smooth immersion.
Then the average value of $\zh$ on $\TT^2$ is at least $\tfrac32$.
\end{thm}

\parit{Proof.}
By the lemma, the average value of $|H|^2$ is at least 4.
Applying the formula and Gauss--Bonnet,
we get the result.
\qeds



\begin{thm}{Theorem}
Suppose $\iota\:\TT^4\to \BB^q$ is a smooth immersion.
Then $\zh(p)\ge 2$ for some point  $p\in\TT^4$.
\end{thm}




Let $g$ be a Riemannian metric on $\TT^n$.
Suppose $n\ge 3$, and $u\:\TT^n\to \RR$ is a positive function.
The scalar curvature of the metric $u^{\frac{4}{n-2}}\cdot g$
can be expressed as 
\[\left(\Sc\cdot u-\tfrac{4{\cdot}(n-1)}{n-2}{\cdot}\Delta u\right)\cdot u^{\frac{n-2}{n+2}}.\]
Recall that \emph{any Riemannian metric $g$ on $\TT^n$ has nonpositive scalar curvature at some point}.
Therefore we get the following.

\begin{thm}{Claim}
Let $g$ be a Riemannian metric on $\TT^n$.
Then, for any positive smooth function $u$ on $\TT^n$, the function 
\[\Sc\cdot u-\tfrac{4{\cdot}(n-1)}{n-2}{\cdot}\Delta u\]
returns a nonpositive value at some point 
\end{thm}


\parit{Proof of the theorem.}
Consider the function $u\:p\mapsto \exp(-|\iota(p)|^2)$.

We will apply the following formula
\[\Delta(\phi\circ f)=\phi'\cdot \Delta f+\phi''\cdot|\nabla f|^2\]
to $f\:p\mapsto \tfrac12\cdot |\iota(p)|^2$ and $\phi\:x\mapsto \exp(-2\cdot x)$; so $u=\phi\circ f$.

Let $\alpha=\measuredangle (H,\iota)$, then $\Delta f=4+|H|\cdot |\iota|\cdot \cos\alpha$ and $|\nabla f|\le|\iota|\cdot \sin\alpha$.
Therefore
\[ \Delta u\le u\cdot[-2\cdot (4+|H|\cdot |\iota|\cdot \cos\alpha)+(2\cdot|\iota|\cdot \sin\alpha)^2].\]
Applying the claim for $n=4$, we get that 
\[\Sc\cdot u-6\cdot \Delta u\]
returns a negative value at some point $p\in \TT^4$.
Applying the formula, we get
\begin{align*}12\cdot \zh(p)
&\ge
\tfrac32\cdot [(|H(p)|-4)^2+16].
\end{align*}
Whence the statement follows.
\qeds




{\sloppy
\printbibliography[heading=bibintoc]
\fussy
}

\end{document}

\section{Extrinsic curvature tensor}

Let $\T$ be a vector space with a scalar product, $\T^n$ will denote
its tensor power of degree $n$, $\S^n(\T)$ and $\Lambda^n(\T)$ will
denote respectively subspace of symmetric and antisymmetric
elements of $\T^n$. The scalar product on $\T$ induces a
scalar product on $\T^n$ and all its subspaces.

The following subspace of $\T^4$
$$\A^4(\T)=\Lambda^4(\T)^\bot\cap \S^2(\Lambda^2(\T))$$
is formed by all possible curvature tensors
on the tangent space $\T$.
In an equivalent way this subspace $\A^4\subset\S^2(\Lambda^2(\T))$
can be described  as the space of all tensors in $\S^2(\Lambda^2(\T))$
satisfying the first Bianchi identity
$$\Rm(X,Y,Z,W)+\Rm(Y,Z,X,W)+\Rm(Z,X,Y,W)=0.$$
In particular, the subspace $\A^4(\T)$ does not depend on the choice of scalar product on $\T$.

\medskip

Let $M\subset\RR^q$ be a smooth submanifold and $s_x\:\S^2(\T_xM)\to \N_x M$ its second fundamental form at $x\in M$;
here $\T M$ and $\N  M$ are tangent and normal bundles over $M$ respectively.
Consider the \emph{extrinsic curvature tensor}
$$\Phi (X,Y,Z,W)\df \langle s(X,Y),s(Z,W)\rangle,$$
here $\Phi$ is a section of $\S^2(\S^2(\T M))$.

Tensor $\Phi$ can be written as
$$\Phi(X,Y,Z,W)=E(X,Y,Z,W)+\tfrac 1 3\cdot(\Rm(X,Z,Y,W)+\Rm(X,W,Y,Z))
\eqno(*)$$ 
where $E$ is the total symmetrization of $\Phi$; that is,
$$E(X,Y,Z,W)=\tfrac 1 3\cdot
(\Phi(X,Y,Z,W)+\Phi(Y,Z,X,W)+\Phi(Z,X,Y,W))\in \S^4(\T),$$
and
$$\Rm(X,Y,Z,W)=\Phi(X,Z,Y,W)-\Phi(X,W,Y,Z)\in \A^4(\T)$$
is the Riemannian curvature tensor of $M$.

Tensor $E$ represents the \emph{purely extrinsic} part $\Phi$.
Note that $E\in \S^4(\T)\z\subset\S^2(\S^2(\T))$.
Tensor $E$ measures wrinkling of the embedding --- the more it is wrinkled the bigger $E$ gets.
Note that
$f(X)=E(X,X,X,X)=|s(X,X)|^2$
is homogeneous polynomial of degree $4$ on $\T$ 
and it describes $E$ completely.
Therefore in some sense, the $E$-tensor is a higher order analog of the metric tensor.


Observe that 
\begin{itemize}
 \item The extrinsic curvature tensor $\Phi$ remembers defines the second fundamental form up to action of orthogonal group on $\N$.
 \item The quantities $h^2$, $\zh$ and $\sigma$ in the claim depend linearly on $\Phi$ and they are invariant with respect to the action of orthogonal group on $\T$.
\end{itemize}

It follows that passing averaging $\Phi$ along the action of orthogonal groups on $\T$ does not change the quantities in the claim.
In particular, we can assume that both $\Rm$ and $E$ are invariant with respecct to orthogonal group.
It follows that $\Rm$ has constant sectional curvature, say $\sigma$ and $E(\vec x,\vec x,\vec x,\vec x)=\zh\cdot |\vec x|^4$ for some $\zh$.
Since $\Phi$ can be expressed linearly from $\Rm$ and $E$, it completely described by two parameters --- $\sigma$ and $\zh$.
Therefore it is sufficient to check the formula for two symmetric embedding with linear independent pairs $\sigma,\zh$.
One such embedding is standard embedding of sphere; in this case we have $\sigma=\zh=|h|^2=1$.
Veronesse immersion gives another example, where direct calculations show that $\zh=\tfrac{n-1}{n+1}$, $\sigma=-\tfrac{n+2}{2\cdot (n+1)}$ and $|h^2|=0$. 

